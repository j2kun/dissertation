\chapter{Introduction}
\label{introduction}

This dissertation contains selected work of the author over the course of
his graduate studies, focusing on the study of computational complexity in new
models. The models studied concern graphs, networks, and the relation between
the global and distributed computation computation of some quantity.

In the first chapter we discuss \emph{anti-coordination} games played on
graphs. These are games in which players (nodes) are connected by edges, and
the players are incentivized to choose strategies that differ from their
neighbors. We study the price of anarchy, which measures the tradeoff between
players acting independently and greedily versus a central planning authority.
We introduce a directed graph generalization which allows one to model both
anti-coordination and coordination incentives. We further prove that the
complexity of computing strategies with certain properties (akin to being a
certain kind of Nash equilibrium) is NP-hard.

In the second chapter we introduce a new model for measuring the complexity of
a combinatorial decision problem called \emph{resilience}. Loosely speaking, an
instance of a problem is resilient if it is satisfiable and remains satisfiable
under small adversarial manipulations. For graph coloring, this corresponds to
a graph which is, say, 3-colorable and remains so even after an adversary adds
an arbitrary edge to the graph. In general, we ask how resilient a problem must
be in order to make finding solutions tractable. Surprisingly, for the previous
example of graph 3-coloring, it remains NP-hard to find a 3-coloring. We
further study the gradient between hardness and tractability for resilient
coloring. We also completely characterize the complexity of resilient boolean
satisfiability: it is either vacuous or NP-hard.

In the third chapter we turn our attention to \emph{MapReduce}, a popular model
of distributed computation which has novel constraints on communication and
space. We first refine an existing theoretical model of MapReduce. Then we
prove a general result on the ability of a two-round MapReduce protocol to
capture all of sublogarithmic space Turing machines. Finally, we prove a
connection between MapReduce, the exponential time hypothesis, and
long-standing open conjectures about complexity hierarchies within simultaneous
time/space-bounded copmlexity classes (TISP). A simplification of this result
is that the exponential time hypothesis implies a hierarchy within linear-space
TISP, which in turn implies a hierarchy within MapReduce for each of the
parameters of interest.

The reader is assumed to be familiar with the basic concepts of computational
complexity---Turing machines, the classes P, NP, TIME, SPACE, the concept of a
complexity hierarchy, basic reductions, etc.---as well as basic concepts in
graph theory and game theory. 
